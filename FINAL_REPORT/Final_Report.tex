\documentclass[11pt]{article}
   
   \usepackage{graphicx}
    \usepackage[breakable]{tcolorbox}
    \usepackage{parskip} % Stop auto-indenting (to mimic markdown behaviour)
    
    \usepackage{iftex}
    \ifPDFTeX
    	\usepackage[T1]{fontenc}
    	\usepackage{mathpazo}

    \else
    	\usepackage{fontspec}
    \fi

    % Basic figure setup, for now with no caption control since it's done
    % automatically by Pandoc (which extracts ![](path) syntax from Markdown).
    \usepackage{graphicx}
    % Maintain compatibility with old templates. Remove in nbconvert 6.0
    \let\Oldincludegraphics\includegraphics
    % Ensure that by default, figures have no caption (until we provide a
    % proper Figure object with a Caption API and a way to capture that
    % in the conversion process - todo).
    \usepackage{caption}
    \DeclareCaptionFormat{nocaption}{}
    \captionsetup{format=nocaption,aboveskip=0pt,belowskip=0pt}

    \usepackage[Export]{adjustbox} % Used to constrain images to a maximum size
    \adjustboxset{max size={0.9\linewidth}{0.9\paperheight}}
    \usepackage{float}
    \floatplacement{figure}{H} % forces figures to be placed at the correct location
    \usepackage{xcolor} % Allow colors to be defined
    \usepackage{enumerate} % Needed for markdown enumerations to work
    \usepackage{geometry} % Used to adjust the document margins
    \usepackage{amsmath} % Equations
    \usepackage{amssymb} % Equations
    \usepackage{textcomp} % defines textquotesingle
    % Hack from http://tex.stackexchange.com/a/47451/13684:
    \AtBeginDocument{%
        \def\PYZsq{\textquotesingle}% Upright quotes in Pygmentized code
    }
    \usepackage{upquote} % Upright quotes for verbatim code
    \usepackage{eurosym} % defines \euro
    \usepackage[mathletters]{ucs} % Extended unicode (utf-8) support
    \usepackage{fancyvrb} % verbatim replacement that allows latex
    \usepackage{grffile} % extends the file name processing of package graphics 
                         % to support a larger range
    \makeatletter % fix for grffile with XeLaTeX
    \def\Gread@@xetex#1{%
      \IfFileExists{"\Gin@base".bb}%
      {\Gread@eps{\Gin@base.bb}}%
      {\Gread@@xetex@aux#1}%
    }
    \makeatother

    % The hyperref package gives us a pdf with properly built
    % internal navigation ('pdf bookmarks' for the table of contents,
    % internal cross-reference links, web links for URLs, etc.)
    \usepackage{hyperref}
    % The default LaTeX title has an obnoxious amount of whitespace. By default,
    % titling removes some of it. It also provides customization options.
    \usepackage{titling}
    \usepackage{longtable} % longtable support required by pandoc >1.10
    \usepackage{booktabs}  % table support for pandoc > 1.12.2
    \usepackage[inline]{enumitem} % IRkernel/repr support (it uses the enumerate* environment)
    \usepackage[normalem]{ulem} % ulem is needed to support strikethroughs (\sout)
                                % normalem makes italics be italics, not underlines
    \usepackage{mathrsfs}
    \usepackage{tabto}

    
    % Colors for the hyperref package
    \definecolor{urlcolor}{rgb}{0,.145,.698}
    \definecolor{linkcolor}{rgb}{.71,0.21,0.01}
    \definecolor{citecolor}{rgb}{.12,.54,.11}

    % ANSI colors
    \definecolor{ansi-black}{HTML}{3E424D}
    \definecolor{ansi-black-intense}{HTML}{282C36}
    \definecolor{ansi-red}{HTML}{E75C58}
    \definecolor{ansi-red-intense}{HTML}{B22B31}
    \definecolor{ansi-green}{HTML}{00A250}
    \definecolor{ansi-green-intense}{HTML}{007427}
    \definecolor{ansi-yellow}{HTML}{DDB62B}
    \definecolor{ansi-yellow-intense}{HTML}{B27D12}
    \definecolor{ansi-blue}{HTML}{208FFB}
    \definecolor{ansi-blue-intense}{HTML}{0065CA}
    \definecolor{ansi-magenta}{HTML}{D160C4}
    \definecolor{ansi-magenta-intense}{HTML}{A03196}
    \definecolor{ansi-cyan}{HTML}{60C6C8}
    \definecolor{ansi-cyan-intense}{HTML}{258F8F}
    \definecolor{ansi-white}{HTML}{C5C1B4}
    \definecolor{ansi-white-intense}{HTML}{A1A6B2}
    \definecolor{ansi-default-inverse-fg}{HTML}{FFFFFF}
    \definecolor{ansi-default-inverse-bg}{HTML}{000000}

    % commands and environments needed by pandoc snippets
    % extracted from the output of `pandoc -s`
    \providecommand{\tightlist}{%
      \setlength{\itemsep}{0pt}\setlength{\parskip}{0pt}}
    \DefineVerbatimEnvironment{Highlighting}{Verbatim}{commandchars=\\\{\}}
    % Add ',fontsize=\small' for more characters per line
    \newenvironment{Shaded}{}{}
    \newcommand{\KeywordTok}[1]{\textcolor[rgb]{0.00,0.44,0.13}{\textbf{{#1}}}}
    \newcommand{\DataTypeTok}[1]{\textcolor[rgb]{0.56,0.13,0.00}{{#1}}}
    \newcommand{\DecValTok}[1]{\textcolor[rgb]{0.25,0.63,0.44}{{#1}}}
    \newcommand{\BaseNTok}[1]{\textcolor[rgb]{0.25,0.63,0.44}{{#1}}}
    \newcommand{\FloatTok}[1]{\textcolor[rgb]{0.25,0.63,0.44}{{#1}}}
    \newcommand{\CharTok}[1]{\textcolor[rgb]{0.25,0.44,0.63}{{#1}}}
    \newcommand{\StringTok}[1]{\textcolor[rgb]{0.25,0.44,0.63}{{#1}}}
    \newcommand{\CommentTok}[1]{\textcolor[rgb]{0.38,0.63,0.69}{\textit{{#1}}}}
    \newcommand{\OtherTok}[1]{\textcolor[rgb]{0.00,0.44,0.13}{{#1}}}
    \newcommand{\AlertTok}[1]{\textcolor[rgb]{1.00,0.00,0.00}{\textbf{{#1}}}}
    \newcommand{\FunctionTok}[1]{\textcolor[rgb]{0.02,0.16,0.49}{{#1}}}
    \newcommand{\RegionMarkerTok}[1]{{#1}}
    \newcommand{\ErrorTok}[1]{\textcolor[rgb]{1.00,0.00,0.00}{\textbf{{#1}}}}
    \newcommand{\NormalTok}[1]{{#1}}
    
    % Additional commands for more recent versions of Pandoc
    \newcommand{\ConstantTok}[1]{\textcolor[rgb]{0.53,0.00,0.00}{{#1}}}
    \newcommand{\SpecialCharTok}[1]{\textcolor[rgb]{0.25,0.44,0.63}{{#1}}}
    \newcommand{\VerbatimStringTok}[1]{\textcolor[rgb]{0.25,0.44,0.63}{{#1}}}
    \newcommand{\SpecialStringTok}[1]{\textcolor[rgb]{0.73,0.40,0.53}{{#1}}}
    \newcommand{\ImportTok}[1]{{#1}}
    \newcommand{\DocumentationTok}[1]{\textcolor[rgb]{0.73,0.13,0.13}{\textit{{#1}}}}
    \newcommand{\AnnotationTok}[1]{\textcolor[rgb]{0.38,0.63,0.69}{\textbf{\textit{{#1}}}}}
    \newcommand{\CommentVarTok}[1]{\textcolor[rgb]{0.38,0.63,0.69}{\textbf{\textit{{#1}}}}}
    \newcommand{\VariableTok}[1]{\textcolor[rgb]{0.10,0.09,0.49}{{#1}}}
    \newcommand{\ControlFlowTok}[1]{\textcolor[rgb]{0.00,0.44,0.13}{\textbf{{#1}}}}
    \newcommand{\OperatorTok}[1]{\textcolor[rgb]{0.40,0.40,0.40}{{#1}}}
    \newcommand{\BuiltInTok}[1]{{#1}}
    \newcommand{\ExtensionTok}[1]{{#1}}
    \newcommand{\PreprocessorTok}[1]{\textcolor[rgb]{0.74,0.48,0.00}{{#1}}}
    \newcommand{\AttributeTok}[1]{\textcolor[rgb]{0.49,0.56,0.16}{{#1}}}
    \newcommand{\InformationTok}[1]{\textcolor[rgb]{0.38,0.63,0.69}{\textbf{\textit{{#1}}}}}
    \newcommand{\WarningTok}[1]{\textcolor[rgb]{0.38,0.63,0.69}{\textbf{\textit{{#1}}}}}
    
    
    % Define a nice break command that doesn't care if a line doesn't already
    % exist.
    \def\br{\hspace*{\fill} \\* }
    % Math Jax compatibility definitions
    \def\gt{>}
    \def\lt{<}
    \let\Oldtex\TeX
    \let\Oldlatex\LaTeX
    \renewcommand{\TeX}{\textrm{\Oldtex}}
    \renewcommand{\LaTeX}{\textrm{\Oldlatex}}
    % Document parameters
    % Document title
    \title{Digitalizing Skin Cancer Detection with
PyTorch}
    
    
    
    
    
% Pygments definitions
\makeatletter
\def\PY@reset{\let\PY@it=\relax \let\PY@bf=\relax%
    \let\PY@ul=\relax \let\PY@tc=\relax%
    \let\PY@bc=\relax \let\PY@ff=\relax}
\def\PY@tok#1{\csname PY@tok@#1\endcsname}
\def\PY@toks#1+{\ifx\relax#1\empty\else%
    \PY@tok{#1}\expandafter\PY@toks\fi}
\def\PY@do#1{\PY@bc{\PY@tc{\PY@ul{%
    \PY@it{\PY@bf{\PY@ff{#1}}}}}}}
\def\PY#1#2{\PY@reset\PY@toks#1+\relax+\PY@do{#2}}

\expandafter\def\csname PY@tok@w\endcsname{\def\PY@tc##1{\textcolor[rgb]{0.73,0.73,0.73}{##1}}}
\expandafter\def\csname PY@tok@c\endcsname{\let\PY@it=\textit\def\PY@tc##1{\textcolor[rgb]{0.25,0.50,0.50}{##1}}}
\expandafter\def\csname PY@tok@cp\endcsname{\def\PY@tc##1{\textcolor[rgb]{0.74,0.48,0.00}{##1}}}
\expandafter\def\csname PY@tok@k\endcsname{\let\PY@bf=\textbf\def\PY@tc##1{\textcolor[rgb]{0.00,0.50,0.00}{##1}}}
\expandafter\def\csname PY@tok@kp\endcsname{\def\PY@tc##1{\textcolor[rgb]{0.00,0.50,0.00}{##1}}}
\expandafter\def\csname PY@tok@kt\endcsname{\def\PY@tc##1{\textcolor[rgb]{0.69,0.00,0.25}{##1}}}
\expandafter\def\csname PY@tok@o\endcsname{\def\PY@tc##1{\textcolor[rgb]{0.40,0.40,0.40}{##1}}}
\expandafter\def\csname PY@tok@ow\endcsname{\let\PY@bf=\textbf\def\PY@tc##1{\textcolor[rgb]{0.67,0.13,1.00}{##1}}}
\expandafter\def\csname PY@tok@nb\endcsname{\def\PY@tc##1{\textcolor[rgb]{0.00,0.50,0.00}{##1}}}
\expandafter\def\csname PY@tok@nf\endcsname{\def\PY@tc##1{\textcolor[rgb]{0.00,0.00,1.00}{##1}}}
\expandafter\def\csname PY@tok@nc\endcsname{\let\PY@bf=\textbf\def\PY@tc##1{\textcolor[rgb]{0.00,0.00,1.00}{##1}}}
\expandafter\def\csname PY@tok@nn\endcsname{\let\PY@bf=\textbf\def\PY@tc##1{\textcolor[rgb]{0.00,0.00,1.00}{##1}}}
\expandafter\def\csname PY@tok@ne\endcsname{\let\PY@bf=\textbf\def\PY@tc##1{\textcolor[rgb]{0.82,0.25,0.23}{##1}}}
\expandafter\def\csname PY@tok@nv\endcsname{\def\PY@tc##1{\textcolor[rgb]{0.10,0.09,0.49}{##1}}}
\expandafter\def\csname PY@tok@no\endcsname{\def\PY@tc##1{\textcolor[rgb]{0.53,0.00,0.00}{##1}}}
\expandafter\def\csname PY@tok@nl\endcsname{\def\PY@tc##1{\textcolor[rgb]{0.63,0.63,0.00}{##1}}}
\expandafter\def\csname PY@tok@ni\endcsname{\let\PY@bf=\textbf\def\PY@tc##1{\textcolor[rgb]{0.60,0.60,0.60}{##1}}}
\expandafter\def\csname PY@tok@na\endcsname{\def\PY@tc##1{\textcolor[rgb]{0.49,0.56,0.16}{##1}}}
\expandafter\def\csname PY@tok@nt\endcsname{\let\PY@bf=\textbf\def\PY@tc##1{\textcolor[rgb]{0.00,0.50,0.00}{##1}}}
\expandafter\def\csname PY@tok@nd\endcsname{\def\PY@tc##1{\textcolor[rgb]{0.67,0.13,1.00}{##1}}}
\expandafter\def\csname PY@tok@s\endcsname{\def\PY@tc##1{\textcolor[rgb]{0.73,0.13,0.13}{##1}}}
\expandafter\def\csname PY@tok@sd\endcsname{\let\PY@it=\textit\def\PY@tc##1{\textcolor[rgb]{0.73,0.13,0.13}{##1}}}
\expandafter\def\csname PY@tok@si\endcsname{\let\PY@bf=\textbf\def\PY@tc##1{\textcolor[rgb]{0.73,0.40,0.53}{##1}}}
\expandafter\def\csname PY@tok@se\endcsname{\let\PY@bf=\textbf\def\PY@tc##1{\textcolor[rgb]{0.73,0.40,0.13}{##1}}}
\expandafter\def\csname PY@tok@sr\endcsname{\def\PY@tc##1{\textcolor[rgb]{0.73,0.40,0.53}{##1}}}
\expandafter\def\csname PY@tok@ss\endcsname{\def\PY@tc##1{\textcolor[rgb]{0.10,0.09,0.49}{##1}}}
\expandafter\def\csname PY@tok@sx\endcsname{\def\PY@tc##1{\textcolor[rgb]{0.00,0.50,0.00}{##1}}}
\expandafter\def\csname PY@tok@m\endcsname{\def\PY@tc##1{\textcolor[rgb]{0.40,0.40,0.40}{##1}}}
\expandafter\def\csname PY@tok@gh\endcsname{\let\PY@bf=\textbf\def\PY@tc##1{\textcolor[rgb]{0.00,0.00,0.50}{##1}}}
\expandafter\def\csname PY@tok@gu\endcsname{\let\PY@bf=\textbf\def\PY@tc##1{\textcolor[rgb]{0.50,0.00,0.50}{##1}}}
\expandafter\def\csname PY@tok@gd\endcsname{\def\PY@tc##1{\textcolor[rgb]{0.63,0.00,0.00}{##1}}}
\expandafter\def\csname PY@tok@gi\endcsname{\def\PY@tc##1{\textcolor[rgb]{0.00,0.63,0.00}{##1}}}
\expandafter\def\csname PY@tok@gr\endcsname{\def\PY@tc##1{\textcolor[rgb]{1.00,0.00,0.00}{##1}}}
\expandafter\def\csname PY@tok@ge\endcsname{\let\PY@it=\textit}
\expandafter\def\csname PY@tok@gs\endcsname{\let\PY@bf=\textbf}
\expandafter\def\csname PY@tok@gp\endcsname{\let\PY@bf=\textbf\def\PY@tc##1{\textcolor[rgb]{0.00,0.00,0.50}{##1}}}
\expandafter\def\csname PY@tok@go\endcsname{\def\PY@tc##1{\textcolor[rgb]{0.53,0.53,0.53}{##1}}}
\expandafter\def\csname PY@tok@gt\endcsname{\def\PY@tc##1{\textcolor[rgb]{0.00,0.27,0.87}{##1}}}
\expandafter\def\csname PY@tok@err\endcsname{\def\PY@bc##1{\setlength{\fboxsep}{0pt}\fcolorbox[rgb]{1.00,0.00,0.00}{1,1,1}{\strut ##1}}}
\expandafter\def\csname PY@tok@kc\endcsname{\let\PY@bf=\textbf\def\PY@tc##1{\textcolor[rgb]{0.00,0.50,0.00}{##1}}}
\expandafter\def\csname PY@tok@kd\endcsname{\let\PY@bf=\textbf\def\PY@tc##1{\textcolor[rgb]{0.00,0.50,0.00}{##1}}}
\expandafter\def\csname PY@tok@kn\endcsname{\let\PY@bf=\textbf\def\PY@tc##1{\textcolor[rgb]{0.00,0.50,0.00}{##1}}}
\expandafter\def\csname PY@tok@kr\endcsname{\let\PY@bf=\textbf\def\PY@tc##1{\textcolor[rgb]{0.00,0.50,0.00}{##1}}}
\expandafter\def\csname PY@tok@bp\endcsname{\def\PY@tc##1{\textcolor[rgb]{0.00,0.50,0.00}{##1}}}
\expandafter\def\csname PY@tok@fm\endcsname{\def\PY@tc##1{\textcolor[rgb]{0.00,0.00,1.00}{##1}}}
\expandafter\def\csname PY@tok@vc\endcsname{\def\PY@tc##1{\textcolor[rgb]{0.10,0.09,0.49}{##1}}}
\expandafter\def\csname PY@tok@vg\endcsname{\def\PY@tc##1{\textcolor[rgb]{0.10,0.09,0.49}{##1}}}
\expandafter\def\csname PY@tok@vi\endcsname{\def\PY@tc##1{\textcolor[rgb]{0.10,0.09,0.49}{##1}}}
\expandafter\def\csname PY@tok@vm\endcsname{\def\PY@tc##1{\textcolor[rgb]{0.10,0.09,0.49}{##1}}}
\expandafter\def\csname PY@tok@sa\endcsname{\def\PY@tc##1{\textcolor[rgb]{0.73,0.13,0.13}{##1}}}
\expandafter\def\csname PY@tok@sb\endcsname{\def\PY@tc##1{\textcolor[rgb]{0.73,0.13,0.13}{##1}}}
\expandafter\def\csname PY@tok@sc\endcsname{\def\PY@tc##1{\textcolor[rgb]{0.73,0.13,0.13}{##1}}}
\expandafter\def\csname PY@tok@dl\endcsname{\def\PY@tc##1{\textcolor[rgb]{0.73,0.13,0.13}{##1}}}
\expandafter\def\csname PY@tok@s2\endcsname{\def\PY@tc##1{\textcolor[rgb]{0.73,0.13,0.13}{##1}}}
\expandafter\def\csname PY@tok@sh\endcsname{\def\PY@tc##1{\textcolor[rgb]{0.73,0.13,0.13}{##1}}}
\expandafter\def\csname PY@tok@s1\endcsname{\def\PY@tc##1{\textcolor[rgb]{0.73,0.13,0.13}{##1}}}
\expandafter\def\csname PY@tok@mb\endcsname{\def\PY@tc##1{\textcolor[rgb]{0.40,0.40,0.40}{##1}}}
\expandafter\def\csname PY@tok@mf\endcsname{\def\PY@tc##1{\textcolor[rgb]{0.40,0.40,0.40}{##1}}}
\expandafter\def\csname PY@tok@mh\endcsname{\def\PY@tc##1{\textcolor[rgb]{0.40,0.40,0.40}{##1}}}
\expandafter\def\csname PY@tok@mi\endcsname{\def\PY@tc##1{\textcolor[rgb]{0.40,0.40,0.40}{##1}}}
\expandafter\def\csname PY@tok@il\endcsname{\def\PY@tc##1{\textcolor[rgb]{0.40,0.40,0.40}{##1}}}
\expandafter\def\csname PY@tok@mo\endcsname{\def\PY@tc##1{\textcolor[rgb]{0.40,0.40,0.40}{##1}}}
\expandafter\def\csname PY@tok@ch\endcsname{\let\PY@it=\textit\def\PY@tc##1{\textcolor[rgb]{0.25,0.50,0.50}{##1}}}
\expandafter\def\csname PY@tok@cm\endcsname{\let\PY@it=\textit\def\PY@tc##1{\textcolor[rgb]{0.25,0.50,0.50}{##1}}}
\expandafter\def\csname PY@tok@cpf\endcsname{\let\PY@it=\textit\def\PY@tc##1{\textcolor[rgb]{0.25,0.50,0.50}{##1}}}
\expandafter\def\csname PY@tok@c1\endcsname{\let\PY@it=\textit\def\PY@tc##1{\textcolor[rgb]{0.25,0.50,0.50}{##1}}}
\expandafter\def\csname PY@tok@cs\endcsname{\let\PY@it=\textit\def\PY@tc##1{\textcolor[rgb]{0.25,0.50,0.50}{##1}}}

\def\PYZbs{\char`\\}
\def\PYZus{\char`\_}
\def\PYZob{\char`\{}
\def\PYZcb{\char`\}}
\def\PYZca{\char`\^}
\def\PYZam{\char`\&}
\def\PYZlt{\char`\<}
\def\PYZgt{\char`\>}
\def\PYZsh{\char`\#}
\def\PYZpc{\char`\%}
\def\PYZdl{\char`\$}
\def\PYZhy{\char`\-}
\def\PYZsq{\char`\'}
\def\PYZdq{\char`\"}
\def\PYZti{\char`\~}
% for compatibility with earlier versions
\def\PYZat{@}
\def\PYZlb{[}
\def\PYZrb{]}
\makeatother


    % For linebreaks inside Verbatim environment from package fancyvrb. 
    \makeatletter
        \newbox\Wrappedcontinuationbox 
        \newbox\Wrappedvisiblespacebox 
        \newcommand*\Wrappedvisiblespace {\textcolor{red}{\textvisiblespace}} 
        \newcommand*\Wrappedcontinuationsymbol {\textcolor{red}{\llap{\tiny$\m@th\hookrightarrow$}}} 
        \newcommand*\Wrappedcontinuationindent {3ex } 
        \newcommand*\Wrappedafterbreak {\kern\Wrappedcontinuationindent\copy\Wrappedcontinuationbox} 
        % Take advantage of the already applied Pygments mark-up to insert 
        % potential linebreaks for TeX processing. 
        %        {, <, #, %, $, ' and ": go to next line. 
        %        _, }, ^, &, >, - and ~: stay at end of broken line. 
        % Use of \textquotesingle for straight quote. 
        \newcommand*\Wrappedbreaksatspecials {% 
            \def\PYGZus{\discretionary{\char`\_}{\Wrappedafterbreak}{\char`\_}}% 
            \def\PYGZob{\discretionary{}{\Wrappedafterbreak\char`\{}{\char`\{}}% 
            \def\PYGZcb{\discretionary{\char`\}}{\Wrappedafterbreak}{\char`\}}}% 
            \def\PYGZca{\discretionary{\char`\^}{\Wrappedafterbreak}{\char`\^}}% 
            \def\PYGZam{\discretionary{\char`\&}{\Wrappedafterbreak}{\char`\&}}% 
            \def\PYGZlt{\discretionary{}{\Wrappedafterbreak\char`\<}{\char`\<}}% 
            \def\PYGZgt{\discretionary{\char`\>}{\Wrappedafterbreak}{\char`\>}}% 
            \def\PYGZsh{\discretionary{}{\Wrappedafterbreak\char`\#}{\char`\#}}% 
            \def\PYGZpc{\discretionary{}{\Wrappedafterbreak\char`\%}{\char`\%}}% 
            \def\PYGZdl{\discretionary{}{\Wrappedafterbreak\char`\$}{\char`\$}}% 
            \def\PYGZhy{\discretionary{\char`\-}{\Wrappedafterbreak}{\char`\-}}% 
            \def\PYGZsq{\discretionary{}{\Wrappedafterbreak\textquotesingle}{\textquotesingle}}% 
            \def\PYGZdq{\discretionary{}{\Wrappedafterbreak\char`\"}{\char`\"}}% 
            \def\PYGZti{\discretionary{\char`\~}{\Wrappedafterbreak}{\char`\~}}% 
        } 
        % Some characters . , ; ? ! / are not pygmentized. 
        % This macro makes them "active" and they will insert potential linebreaks 
        \newcommand*\Wrappedbreaksatpunct {% 
            \lccode`\~`\.\lowercase{\def~}{\discretionary{\hbox{\char`\.}}{\Wrappedafterbreak}{\hbox{\char`\.}}}% 
            \lccode`\~`\,\lowercase{\def~}{\discretionary{\hbox{\char`\,}}{\Wrappedafterbreak}{\hbox{\char`\,}}}% 
            \lccode`\~`\;\lowercase{\def~}{\discretionary{\hbox{\char`\;}}{\Wrappedafterbreak}{\hbox{\char`\;}}}% 
            \lccode`\~`\:\lowercase{\def~}{\discretionary{\hbox{\char`\:}}{\Wrappedafterbreak}{\hbox{\char`\:}}}% 
            \lccode`\~`\?\lowercase{\def~}{\discretionary{\hbox{\char`\?}}{\Wrappedafterbreak}{\hbox{\char`\?}}}% 
            \lccode`\~`\!\lowercase{\def~}{\discretionary{\hbox{\char`\!}}{\Wrappedafterbreak}{\hbox{\char`\!}}}% 
            \lccode`\~`\/\lowercase{\def~}{\discretionary{\hbox{\char`\/}}{\Wrappedafterbreak}{\hbox{\char`\/}}}% 
            \catcode`\.\active
            \catcode`\,\active 
            \catcode`\;\active
            \catcode`\:\active
            \catcode`\?\active
            \catcode`\!\active
            \catcode`\/\active 
            \lccode`\~`\~ 	
        }
    \makeatother

    \let\OriginalVerbatim=\Verbatim
    \makeatletter
    \renewcommand{\Verbatim}[1][1]{%
        %\parskip\z@skip
        \sbox\Wrappedcontinuationbox {\Wrappedcontinuationsymbol}%
        \sbox\Wrappedvisiblespacebox {\FV@SetupFont\Wrappedvisiblespace}%
        \def\FancyVerbFormatLine ##1{\hsize\linewidth
            \vtop{\raggedright\hyphenpenalty\z@\exhyphenpenalty\z@
                \doublehyphendemerits\z@\finalhyphendemerits\z@
                \strut ##1\strut}%
        }%
        % If the linebreak is at a space, the latter will be displayed as visible
        % space at end of first line, and a continuation symbol starts next line.
        % Stretch/shrink are however usually zero for typewriter font.
        \def\FV@Space {%
            \nobreak\hskip\z@ plus\fontdimen3\font minus\fontdimen4\font
            \discretionary{\copy\Wrappedvisiblespacebox}{\Wrappedafterbreak}
            {\kern\fontdimen2\font}%
        }%
        
        % Allow breaks at special characters using \PYG... macros.
        \Wrappedbreaksatspecials
        % Breaks at punctuation characters . , ; ? ! and / need catcode=\active 	
        \OriginalVerbatim[#1,codes*=\Wrappedbreaksatpunct]%
    }
    \makeatother

    % Exact colors from NB
    \definecolor{incolor}{HTML}{303F9F}
    \definecolor{outcolor}{HTML}{D84315}
    \definecolor{cellborder}{HTML}{CFCFCF}
    \definecolor{cellbackground}{HTML}{F7F7F7}
    
    % prompt
    \makeatletter
    \newcommand{\boxspacing}{\kern\kvtcb@left@rule\kern\kvtcb@boxsep}
    \makeatother
    \newcommand{\prompt}[4]{
        \ttfamily\llap{{\color{#2}[#3]:\hspace{3pt}#4}}\vspace{-\baselineskip}
    }
    

    
    % Prevent overflowing lines due to hard-to-break entities
    \sloppy 
    % Setup hyperref package
    \hypersetup{
      breaklinks=true,  % so long urls are correctly broken across lines
      colorlinks=true,
      urlcolor=urlcolor,
      linkcolor=linkcolor,
      citecolor=citecolor,
      }
    % Slightly bigger margins than the latex defaults
    
    \geometry{verbose,tmargin=1in,bmargin=1in,lmargin=1in,rmargin=1in}
    
    

\begin{document}
    
    \maketitle
    
    

    






\begin{figure}
  \includegraphics[width=\linewidth]{images/melanoma.jpg}
  \caption {Picture of Melanoma.}
  \label{fig:melanoma}
\end{figure}
Figure \ref{fig:melanoma} Picture of Melanoma.

    \hypertarget{introduction}{%
\section{Introduction}\label{introduction}}

\qquad Skin cancer is the most common form of cancer, with 1 in 5 Americans
developing it by the time they reach 70 years old. More then 2 people die of
skin cancer in the US every hour\textsuperscript{\hyperlink{[1]}{[1]}}. Early detection is key to
saving lives, with the early detection 5 year survival rate of skin
cancer being 99\%\textsuperscript{\hyperlink{[1]}{[1]}}. Dermatologists have to look at patients one by
one, and must assess by eye whether or not a blemish is malignant or
benign. Dermatologists have around a 66\% accuracy rate in assessing
752 different skin diseases, while Covolutional Neural Networks, such as the one detailed in
\emph{Dermatologist-level classification of skin cancer with deep neural
networks} published in Nature have achieved greater accuracy levels
then dermatologists, around 72.1\%\textsuperscript{\hyperlink{[2]}{[2]}}.

\qquad By converting cancer detection to easily deployable software, you could
allow people to get accurate cancer testing at home, saving resources
and time. By making cancer detection more accesible, people would be
more likely to get tested, saving lives in the process. Below I will
detail my process and results from a melanoma (malignant), nevus
(benign) , and seborrheic keratosis (benign) detector model using
Convolutional Neural Networks.

    \hypertarget{process}{%
\section{Process}\label{process}}

    \hypertarget{dataset}{%
\subsubsection{Dataset}\label{dataset}}

\qquad The dataset that was used was retrieved from Udacity's Dermatologist AI
mini project. The training set contains 374 pictures of labelled images
of melanoma(skin cancer), 1372 labelled image of nevus(a benign blemish), and 254 labelled images of
seborrheic keratosis(a benign blemish). The validation and testing set each contain about
20\% of the total images in the training set with similiar ratios of
melanoma to nevus to seborrheic keratosis. The image was loaded into
PyTorch using the ImageFolder class.

\hypertarget{model-architecture}{%
\subsubsection{Model Architecture}\label{model-architecture}}

\qquad The model architecture that was used was the EfficientNet model detailed
in the article \emph{EfficientNet: Rethinking Model Scaling for
Convolutional Neural Networks} by Mingxing Tan, and Quoc V. Le\textsuperscript{\hyperlink{[3]}{[3]}}.
This architecture was chosen because it is among the highest ranking
models in the \emph{ImageNet} contest\textsuperscript{\hyperlink{[4]}{[4]}} and is extremely accurate.
The model architecture was used completly unaltered except for the last
dropout and linear layer. The last dropout layer was upped to 50\%
probability, and the last linear layer was adjusted to three output
nodes to fit the classifying task. The weights were initialized as the
\emph{ImageNet} competition weights, which was provided in PyTorch by
GitHub user lukemelas\textsuperscript{\hyperlink{[5]}{[5]}}. Gradient descent was enabled for all
layers, so the weights of the layers were all altered during the
training process to better fit the task of classifying various skin
lesions.


\begin{figure}
  \includegraphics[width=\linewidth]{images/efficient_net.png}
  \caption {EfficientNet Model Architecture[6].}
  \label{fig:EffNet}
\end{figure}
Figure \ref{fig:EffNet} EfficientNet Model Architecture\textsuperscript{\hyperlink{[6]}{[6]}}.


\hypertarget{optimizer}{%
\subsubsection{Optimizer}\label{optimizer}}

\qquad The optimizer that was selected was the Adam optimizer, as detailed in
the article \emph{Adam: A Method for Stochastic Optimization} by
Diederik P. Kingma, and Jimmy Ba\textsuperscript{\hyperlink{[7]}{[7]}}. The Adam optimizer was chosen
because it consistently performs the highest out of all the PyTorch
optimizers, providing fast and effective training.


\begin{figure}
  \includegraphics[width=\linewidth]{images/adam.png}
  \caption {Performance of Adam Optimizer vs Other Common Optimizers.}
  \label{fig:Adam}
\end{figure}
Figure \ref{fig:Adam} Performance of Adam Optimizer vs Other Common Optimizers\textsuperscript{\hyperlink{[8]}{[8]}}.


\hypertarget{loss-function}{%
\subsubsection{Loss Function}\label{loss-function}}

\qquad The loss function that was chosen was Cross Entropy loss. Cross Entropy
loss was chosen because it is simple to use and effective, with high
performance. Data Augmentation using the PyTorch.transforms
functions, all of the images were cropped to 299x299 as it allowed the
model to run at a reasonable speed while attaining good results. All of
the images were normalized according to the \emph{ImageNet} values.
\hypertarget{Data Augmentation}{%
\subsubsection{Data Augmentation}\label{Data Augmentation}}

\qquad The training set was augmented using random rotations, vertical
flips, horizontal flips, and crops from the default PyTorch transforms
library. Other transforms were tried, such as no augmentation, random
affline, gaussian blur, increased noise, and random perspective; however, all
resulted in worse accuracy than the augmentation that was used.


\begin{figure}
  \includegraphics[width=\linewidth]{images/data_aug.png}
  \caption {Augmented Data Images}
  \label{fig:Aug_Img}
\end{figure}
Figure \ref{fig:Aug_Img} Resulting Images from Applied Data Augmentation.


\hypertarget{number-of-epochs-and-early-stopping}{%
\subsubsection{Number of Epochs and Early
Stopping}\label{number-of-epochs-and-early-stopping}}

\qquad I implemented early stopping functionality into my model to ensure that
no time was wasted during training. By stopping the training loop after
6 epochs of no improvment on validation loss, I ensured that I was not
wasting time training my model just to overfit the data set.
Additionally, the model was only saved if it achieved a new best score in
validation loss. Implementing early stopping also allowed me to set a
large amount of epochs and not have to worry about my code running for
too long, as I knew it would just keep going until it was not improving
anymore.

    \hypertarget{results}{%
\section{Results}\label{results}}

    \hypertarget{final-validation-loss-and-test-accuracy}{%
\subsubsection{Final Validation Loss and Test
Accuracy}\label{final-validation-loss-and-test-accuracy}}

\qquad The final model ended up training for 10 epochs and achieved a validation loss
of 0.513393, which was the lowest validation loss of any model tested.
The testing accuracy of the model ended up being 76\%, correctly
classifying 456/600 images. This is notably higher then the
dermatologist rating of 66\%, though with notably fewer classes (752 vs
3).  

\hypertarget{ROC Curve}{%
\subsubsection{ROC Curve}\label{ROC Curve}}

\qquad  The ROC Curve of the model points out a glaring
fault. The model does exceedingly well at correctly classfying benign
lesions from melanoma as a binary task (task\_1) but struggles with
classifying melanoma out of the 3 possible classes, i.w.
melanoma, seborrheic keratosis, and nevus (task\_2). This is concerning
considering malignant lesions pose the greatest harm to health, and
may results in death if missed. The reasons for this discrepancy
are most likely due to the abudance of pictures of benign lesions
compared to malignant lesions in the training data. There are only 372
pictures of melanoma in the training data, while there are over 1500
pictures of benign lesions (nevus and seborrheic keratosis). This most
likely lead to the convulational layer weights becoming overfitted to
classifying benign lesions, while doing a poor job at classifying
malignant ones. I attempted to add more dropout layers to the
EfficientNet model, however it lead to a drastic drop in accuracy. This
leads me to believe that the best solution would be to implement more
data into the experiment. There are millions of labelled images of
Melanoma available from sources such as the \emph{Kaggle SIIM-ISIC
Melanoma Classification} competition\textsuperscript{\hyperlink{[9]}{[9]}};  however my current
equipment setup is not powerful enough to deal with that amount of data.
This will be something I will have to look to implement in the future, when I have more resources available.


\begin{figure}
  \includegraphics[width=10cm]{images/eff_net_base_roc.png}
  \caption{ROC Curve Generated by the Model (Task\_1 = classfying benign lesions
from melanoma as a binary task, Task\_2 = classifying melanoma out of
the 3 possible classes, melanoma, nevus, and seborrheic keratosis).}
  \label{fig:roc_curve}
\end{figure}

Figure \ref{fig:roc_curve} ROC Curve Generated by the Model (Task\_1 = classfying benign lesions
from melanoma as a binary task, Task\_2 = classifying melanoma out of
the 3 possible classes, melanoma, nevus, and seborrheic keratosis).


\hypertarget{confusion-matrix}{%
\subsubsection{Confusion Matrix}\label{confusion-matrix}}

\qquad A confusion matrix represents the amount of correctly labelled postives
vs correctly labelled negatives vs incorrectly labelled positives vs
incorrectly labelled negatives. Positives in this case means the patient
has melanoma and negative means the patient has a benign lesion. I set
the threshold for considering a skin lesion to be melanoma at 30\% (as in if the
model believes with a 30\% probability the skin lesion is melanoma, it
counts it as positive). I chose this because in general, it is safer to
misdiagnose someone with melanoma, then to miss a melanoma diagnosis
completly, as the latter could lead to death while the former might
slightly inconvenience someone.  With
those parameters, I received 71\% correctly labelled positives (correctly labelled as
melanoma) and 81\% correct labelled negatives (correctly labelled as benign skin lesions).
These are good results,exceeding the average
dermatologist rate of 66\%\textsuperscript{\hyperlink{[1]}{[1]}}.


\begin{figure}
  \includegraphics[width=\linewidth]{images/eff_net_0.3_confusion.png}
  \caption{Shows the Fraction of Correctly Labelled Malignant Lesions and Benign
Lesions.}
  \label{fig:confusion_matrix}
\end{figure}

Figure \ref{fig:confusion_matrix} Shows the Fraction of Correctly Labelled Malignant Lesions and Benign
Lesions.

\newpage

    \hypertarget{reflection}{%
\section{Reflection}\label{reflection}}

\qquad This project was an extremely rewarding experience. It allowed me to see
the potential impact that Neural Networks and AI could have on the
future. Digitilizing patient care could help speed up the process, make
it more accurate, reduce the costs, and make top quality healthcare
accessible to everyone. While the results of my trained model were not
perfect, they are certainly promising, and I am extremely excited to
continue working with Machine Learning. My passion is AI, and I will
continue working and improving my skills to help make the world a better place.



\newpage



    \hypertarget{bibliography}{%
\section{Bibliography}\label{bibliography}}


\hypertarget{[1]}{[1] Skin Cancer Facts \& Statistics. (2020, July 10). Retrieved July
15, 2020, from
https://www.skincancer.org/skin-cancer-information/skin-cancer-facts/}

\hypertarget{[2]}{[2] Esteva, A., Kuprel, B., Novoa, R. A., Ko, J., Swetter, S. M.,
Blau, H. M., \& Thrun, S. (2017). Dermatologist-level classification of
skin cancer with deep neural networks. Nature, 542(7639), 115-118.
doi:10.1038/nature21056}

\hypertarget{[3]}{[3] Tan, M., Le, Q. V. (2019). EfficientNet: Rethinking Model Scaling
for Convolutional Neural Networks. arXiv:1905.11946 {[}cs.LG{]}}


\hypertarget{[4]}{[4] Melas, L. (n.d.). Lukemelas/EfficientNet-PyTorch. Retrieved July
19, 2020, from https://github.com/lukemelas/EfficientNet-PyTorch}


\hypertarget{[5]}{[5] Papers with Code - ImageNet Leaderboard. (n.d.). Retrieved July
19, 2020, from
https://paperswithcode.com/sota/image-classification-on-imagenet}


\hypertarget{[6]}{[6] Tan, M., \& Le, Q. V. (2019, May 29). EfficientNet: Improving
Accuracy and Efficiency through AutoML and Model Scaling. Retrieved July
20, 2020, from
https://ai.googleblog.com/2019/05/efficientnet-improving-accuracy-and.html}


\hypertarget{[7]}{[7] Kingma, D. P., Ba, J. (2014). Adam: A Method for Stochastic
Optimization. arXiv:1412.6980v9 {[}cs.LG{]}}

\hypertarget{[8]}{[8] Mack, D. (2018, April 10). How to pick the best learning rate for
your machine learning project. Retrieved July 20, 2020, from
https://medium.com/octavian-ai/which-optimizer-and-learning-rate-should-i-use-for-deep-learning-5acb418f9b2cs.LG}

\hypertarget{[9]}{{[}9{]}SIIM-ISIC Melanoma Classification. (2020). Retrieved July 20,
2020, from https://www.kaggle.com/c/siim-isic-melanoma-classification}



    % Add a bibliography block to the postdoc
    
    
    
\end{document}
